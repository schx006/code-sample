\usepackage[utf8]{inputenc}
% permet à l'éditeur de texte d'interpréter les
% caractères supportés par l'encodage utf8
\usepackage[T1]{fontenc}
% permet la lecture des caractères spéciaux
\usepackage[french]{babel}
% le texte généré suivra la convention
% de la langue choisie (français)
%\usepackage{color}
% permet l'utilisation de la couleur dans le texte

\documentclass[french, 12pt, twoside, a4paper]{article}
% classe du document :
% article (à utiliser pour des textes courts)
% d'autres classes sont :
% book, report, letter, beamer…
% options :
% french (langue utilisée pour la typographie)
% 12pt (taille de la police, par défaut 10pt)
% twoside (impression recto-verso)
% a4paper (format du papier A4)

