\documentclass[frenchb, 12pt, twoside, a4paper]{article}
% Définition du document
% Classes : book, report, thesis, article, letter, beamer…
% Options :
%   frenchb   (module linguistique utilisée pour la typographie)
%   12pt      (taille de la police, par défaut 10pt)
%   twoside   (impression recto-verso)
%   a4paper   (format du papier : A4)


%% PRÉAMBULE
\usepackage[utf8]{inputenc}      % Encodage utf-8
\usepackage[T1]{fontenc}         % Règles de césure avec les caractères accentués
\usepackage{babel}               % Conventions typographiques linguistiques française
                                 % inutile de préciser l'option [frenchb] si celle-ci
                                 % est déjà déclarée par "\documentclass"
\usepackage{lmodern}             % Police standard : "Latin Modern"
                                 % alternative à la police d'origine : Computer Modern
\usepackage[top=2.5cm, bottom=2.5cm, 
  left=2cm, right=2cm]{geometry} % Définition des marges
%\usepackage{color}               % Mise en couleur du texte
%\usepackage{graphicx}            % Insertion d'images (format jpeg ou png uniquement)
%\usepackage{amsmath}             % Extension pour l'environnement "Mathématiques"
%\usepackage{amssymb}             % Extension pour les symboles mathématiques
%\usepackage{lipsum}              % Texte factice "Lorem Ipsum"


% Titre
\title{Titre du document}

% Auteur(s)
\author{
  Prénom \textsc{Nom de l'auteur} 
  \thanks{email@xs-net.io} \\           % Adresse mail en "pied de page"
  Fonction de l'auteur \\
  Établissement (par exemple) \\
%  \texttt{email@xs-net.io}              % Alternative pour l'adresse mail
%  \and                                  % Ajout d'un autre auteur
  }

% Date
\date{\today}                           % document daté du jour de la compilation


%% FIN DU PRÉAMBULE
%% DÉBUT DU DOCUMENT
\begin{document}

%% Page de garde
\maketitle                              % insertion du titre

%% Sommaire -- Penser à compiler deux fois
{
\renewcommand{\contentsname}{Sommaire}  % lorsqu'elle est placée en début de document, 
                                        % la "Table des matières" s'appelle "Sommaire"
\tableofcontents                        % insertipon du sommaire
}


%% Résumé
% Changer le titre du résumé
%\renewcommand{\abstractname}{\Large{}\textbf{Résumé}}
% Résumé
\begin{abstract}
Court texte de présentation.
\end{abstract}
% Classe report : sur une page à part
% Classe article : sur la page de garde (si pas de newpage)

% Si besoin : saut de page et initialisation du compteur de pages
%\clearpage\setcounter{page}{2}


\section*{Introduction}
% '*' permet de ne pas numéroter l'entrée
% dans la table des matères

Ici, c'est l'intro…

\section{Un chapitre}

%\include{myFile.tex}
% importation d'un fragment du document depuis un autre fichier
% les fichiers appelés doivent être dans le même répertoire que
% celui qui l'appelle


\subsection{Un sous-chapitre}
…
\paragraph{Sous-partie de sous-chapitre}
…


\subsection{Autre sous-chapitre}
…


%% FIN DU DOCUMENT
\end{document}
