\documentclass[frenchb, 12pt, twoside, a4paper]{article}
% Définition du document
% Classes : book, report, thesis, article, letter, beamer…
% Options :
%   frenchb   (module linguistique utilisée pour la typographie)
%   12pt      (taille de la police, par défaut 10pt)
%   twoside   (impression recto-verso)
%   a4paper   (format du papier : A4)


%% Préambule

\usepackage[utf8]{inputenc}      % Encodage utf-8
\usepackage[T1]{fontenc}         % Règles de césure avec les caractères accentués
\usepackage{babel}               % Conventions typographiques linguistiques française
                                 % inutile de préciser l'option [frenchb] car celle-ci
                                 % est déjà déclarée par "\documentclass"
\usepackage{lmodern}             % Police standard : "Latin Modern"
                                 % alternative à la police d'origine : Computer Modern
\usepackage[top=2.5cm, bottom=2.5cm, 
  left=2cm, right=2cm]{geometry} % Définition des marges
%\usepackage{color}               % Mise en couleur du texte
%\usepackage{graphicx}            % Insertion d'images (format jpeg ou png uniquement)


% Titre
\title{Titre du document \LaTeX{}}

% Auteur
\author{Xavier, aka \textsc{schxOO6} \thanks{schx006@xs-net.io} \\
Geek}

% Date
\date{\today}                           % document daté du jour de la compilation


%% Fin du préambule

\begin{document}

%% Page de garde
\maketitle                              % insertion du titre

%% Sommaire -- Penser à compiler deux fois
{
\renewcommand{\contentsname}{Sommaire}  % lorsqu'elle est placée en début de document, 
                                        % la "Table des matières" s'appelle "Sommaire"
\tableofcontents                        % insertipon du sommaire
}


%% Résumé
% Changer le titre du résumé
%\renewcommand{\abstractname}{\Large{}\textbf{Résumé}}
% Résumé
\begin{abstract}
Court texte de présentation : \\
Ce document permet de faire différents essai d'utilisation
du langage \LaTeX{}.

Il a aussi pour objectif de tester le compilateur \LaTeX{}
dans différents environnements 
\emph{Mac OS\textsuperscript{®}, GNU Linux, MS Windows\textsuperscript{®}},
grâce à la disponibilité des fichiers sources sur le 
\enquote{repository Github\textsuperscript{®}}.
\end{abstract}
% Classe report :  sur une page à part
% Classe article : sur la page de garde (si pas de "newpage"")

% Si besoin : saut de page et initialisation du compteur de pages
%\clearpage\setcounter{page}{2}


\section*{Introduction}
% '*' permet de ne pas numéroter l'entrée
% dans la table des matères

% L'introduction → fichier "intro.tex"
\insclude{intro.tex}

\section{Un chapitre}
…

\subsection{Un sous-chapitre}
…
\paragraph{Sous-partie de sous-chapitre}
…

\subsection{Autre sous-chapitre}
…

\section*{Conclusion}
% La Conclusion → fichier "conclusion.tex"
\insclude{conclusion.tex}

\end{document}
