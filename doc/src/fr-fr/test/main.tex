\documentclass[frenchb, 12pt, twoside, a4paper]{article}
% Définition du document
% Classes : book, report, thesis, article, letter, beamer…
% Options :
%   frenchb   (module linguistique utilisée pour la typographie)
%   12pt      (taille de la police, par défaut 10pt)
%   twoside   (impression recto-verso)
%   a4paper   (format du papier : A4)


%% PRÉAMBULE
\usepackage[utf8]{inputenc}      % Encodage utf-8
\usepackage[T1]{fontenc}         % Règles de césure avec les caractères accentués
\usepackage{babel}               % Conventions typographiques linguistiques française
                                 % inutile de préciser l'option [frenchb] si celle-ci
                                 % est déjà déclarée par "\documentclass"
\usepackage{lmodern}             % Police standard : "Latin Modern"
                                 % alternative à la police d'origine : Computer Modern
\usepackage{subfiles}            % Permets de scinder chaque partie dans un fichier…
\usepackage[top=2.5cm, bottom=2.5cm, 
  left=2cm, right=2cm]{geometry} % Définition des marges
%\usepackage{color}               % Mise en couleur du texte
%\usepackage{graphicx}            % Insertion d'images (format jpeg ou png uniquement)
%\grapicspath{images/}            % Sous-répertoire pour les images : "images/"
\usepackage{amsmath}             % Extension pour l'environnement "Mathématiques"
\usepackage{amssymb}             % Extension pour les symboles mathématiques
\usepackage{comment}             % Permet de "commenter" des blocs entiers…
\usepackage{lipsum}              % Texte factice "Lorem Ipsum"


% Titre
\title{Titre du document \LaTeX{}}

% Auteur(s)
\author{
  Jean \textsc{Tartempion}
  Geek \\
  France \\
  \texttt{email@xs-net.io} \\           % Adresse mail (police "teletype")
  \and
  John \textsc{Doe} \\
  \thanks{Financé par Anonymous.} \\    % Note de bas de page de "remerciement"
  Hacker \\
  Nulle-Part
  }

% Date
\date{\today}                           % document daté du jour de la compilation


%% FIN DU PRÉAMBULE
%% DÉBUT DU DOCUMENT
\begin{document}

%% Page de garde
\maketitle                              % insertion du titre

%% Sommaire -- Penser à compiler deux fois
{
\renewcommand{\contentsname}{Sommaire}  % lorsqu'elle est placée en début de document, 
                                        % la "Table des matières" s'appelle "Sommaire"
\tableofcontents                        % insertipon du sommaire
}


%% Résumé
% Changer le titre du résumé
%\renewcommand{\abstractname}{\Large{}\textbf{Résumé}}
% Résumé
\begin{abstract}
  Court texte de présentation~: \\
  Ce document permet de faire différents essai d'utilisation
  du langage \LaTeX{}.
  
  Il a aussi pour objectif de tester le compilateur \LaTeX{}
  dans différents environnements 
  \emph{Mac OS\textsuperscript{®}, GNU Linux, MS Windows\textsuperscript{®}},
  grâce à la disponibilité des fichiers sources sur le 
  \enquote{repository Github\textsuperscript{®}}.
  
  Bien sûr, ce document n'a ni queue, ni tête~! \\
  \Large{🙃}
\end{abstract}
% Classe report :  sur une page à part
% Classe article : sur la page de garde (si pas de "newpage")

% Si besoin : saut de page et initialisation du compteur de pages
%\clearpage\setcounter{page}{2}


\section*{Introduction}
% '*' permet de ne pas numéroter l'entrée
% dans la table des matères

% L'introduction → fichier "intro.tex"
\subfile{sections/intro}

\section{Un chapitre}
\subfile{sections/section1}

\begin{comment}
\subsection{Un sous-chapitre}
…
\paragraph{Exemple de fonction récursive}

La fonction emph{factorielle} est l'exemple typique de la fonction récursive. \\
Voici sa définition~:

\begin{displaymath}
	\forall n \in \mathbb{N}^{*} : n! = n \times (n - 1) et 1! = 1
\end{displaymath}

\subsection{Autre sous-chapitre}
…
\end{comment}

\section*{Conclusion}
% La Conclusion → fichier "conclusion.tex"
\subfile{sections/conclusion}

%% FIN DU DOCUMENT
\end{document}
